\documentclass[11pt,a4paper]{article}
\usepackage[T1]{fontenc}
\usepackage{isabelle,isabellesym}
\usepackage{amsmath}
\usepackage{amssymb}
\usepackage{geometry}
\geometry{
  left=3cm,
  right=3cm,
  top=2cm,
  bottom=4cm,
  bindingoffset=5mm
}

% this should be the last package used
\usepackage{pdfsetup}

% do not indent paragraphs
\setlength\parindent{0pt}

% urls in roman style, theory text in math-similar italics
\urlstyle{rm}
\isabellestyle{it}

\definecolor{keyword}{RGB}{0,153,102}
\definecolor{command}{RGB}{0,102,153}
\isabellestyle{tt}
\renewcommand{\isacommand}[1]{\textcolor{command}{\textbf{#1}}}
\renewcommand{\isakeyword}[1]{\textcolor{keyword}{\textbf{#1}}}


\begin{document}

\title{Formalization of a Weak Spectroscopy Game deciding Behavioural Equivalences}
\author{Lisa Annett Barthel \and Benjamin Bisping \and Leonard Moritz Hübner \and Caroline Lemke \and Karl Parvis Philipp Mattes \and Lenard Mollenkopf}
\maketitle

\begin{abstract}
\noindent We provide an Isabelle/HOL formalization of Bisping and Jansen's \cite{bisping2023lineartimebranchingtime} weak spectroscopy game, which can be used to decide
a range of behavioural equivalences simultaneously, spanning from stability-respecting branching bisimilarity to weak trace equivalence.
We outline the proof of correctness by relating distinguishing sublanguages of Hennessy-Milner Logic and attacker-winning budgets by an eight-dimensional 
measure of formula expressiveness (cf.\cite[Abstract]{bisping2023lineartimebranchingtime}). We provide small but necessary adaptations to the game and the proof.
\end{abstract}

\tableofcontents

% include generated text of all theories
\newpage
\section{Introduction} %todo reference to expose
Verification and asking wether a model fulfills its specification or if a program can be replaced with one that has the same behaviour are core problems of reactive systems and programming.
For this we have to get an idea of what same behavior for processes actually means.
One possibility are games where the winner of the game corresponds to the behavioural equivalence of processes.
Alternatively, we could use a modal logic, known as Hennessy-Milner Logic (HML), not only to express the specification but also to build formulas that distinguish processes and 
thereby characterize behavioural equivalences.
These techniques for checking whether two processes have the same behaviour may also be combined.
\\\\
Previously, it was only possible to decide equivalence problems individually, but recently there have been ideas of deciding many of these problems at once.
Therefore, Bisping and Jansen \cite{bisping2023lineartimebranchingtime} included a measure of expressiveness for HML$_\text{{SRBB}}$ formulas as eight-dimensional vectors.
These vectors are added as costs to the moves of an extended delay bisimulation game such that the following property is obtained: 
The attacker wins a play with a certain initial energy $e$ if and only if there is a formula that distinguishes the corresponding processes with a price less than or equal to $e$.
Then the initial energy and the price of a formula encode the satisfied behavioural equivalences.
Therefore it is possible to decide for a whole spectrum of behavioural equivalences at the same time which of them apply.
\\\\
We formalize the eight-dimensional weak spectroscopy game, 
which ``can be used to decide a wide array of behavioral equivalences between stability-respecting branching bisimilarity and weak trace equivalence in one go''\cite{bisping2023lineartimebranchingtime}.
We then outline the proof of the correspondence between ``attacker-winning energy budgets and distinguishing sublanguages of Hennessy-Milner logic characterized by eight dimensions of formula expressiveness''.
With our formalization, we try to follow \cite{bisping2023lineartimebranchingtime} as closely as possible in how we formalize the spectroscopy game, HML and their correspondence.
In doing so, we point out deviations in our formalization from and small corrections of the paper.
\\\\
First, we formalize the term LTS and the different kinds of steps.
Afterwards we describe our formalization of HML and a subset of HML, which we denote HML$_\text{srbb}$.
Within this HML section, we introduce several implications and equivalences that we expect to hold for HML formulas.
In the following sections, we present our formalization of energies as a data type that should fulfill certain properties and a price function for formulas.
Before we outline a proof of correctness for the correspondence between the results of the spectroscopy game and distinguishing formulas we formalize spectroscopy game and its basis in the form of energy games.
Following these fundamentals, we state our formalization of the theorem $1$ of \cite{bisping2023lineartimebranchingtime}, that ``relates attacker-winning energy budgets and distinguishing sublanguages of Hennessy-Milner logic''. 
Based on the proof in \cite{bisping2023lineartimebranchingtime} we show this theorem through three lemmas. 
The first lemma states that given a distinguishing formula, the attacker is able to win the corresponding spectroscopy game, which we prove by induction.
Using the same technique, we prove the second lemma, which claimes that if the attacker wins the spectroscopy game with an initial energy $e$, then there exists a (strategy) formula with a price less than or equal to $e$.
Afterwards, the third lemma completes this cycle by stating that if there is a (strategy) formula $\phi$, then $\phi$ distingusihes the corresponding processes.
Finally, we discuss the errors and inaccuracies we found in the paper and thus present our contributions to the paper\cite{bisping2023lineartimebranchingtime}.
\input{LTS}
\input{HML}
\input{HML_SRBB}
\input{Energy}
\input{Expressiveness_Price}
\input{Branching_Bisimilarity}
\input{Energy_Games}
\input{Spectroscopy_Game}
\input{Silent_Step_Spectroscopy}
\input{Distinction_Implies_Winning_Budgets}
\input{Strategy_Formulas}
\section{Contributions to the Paper}

\begin{itemize}
\item In the definition of the weak spectroscopy game in Section \ref{specmoves} the moves were changed: 
\\
The finishing move was removed. This move adjusted the cost of an empty immediate conjunction to be zero.
This is inconsistent with the definition of the expressiveness price. Also the move is not needed. In the 
case of an empty set $Q$ at an attacker (immediate) position the attacker can move to the corresponding 
defender conjunction position by first going delaying. These two moves combined actually have zero costs.
\\
The move empty\_stbl\_conj\_answer (moving from adefender stable conjunction position to a defender conjunction
position if $Q$ is empty) with costs $e_4$ was added. This is necessary for the proof of lemma 
winning\_budget\_implies\_strategy\_formula in section \ref{derivation:lemma3}. Before 
$e \in Win_a (p, \emptyset)_d^s $ was true for any energy (not equal to the defender\_win\_level), in particular $\bold{0}$. 
Since an empty stable conjuction has costs of $e_4$ no strategy formula with an appropriate price could be constructed. 
With the added move $e_4 \leq e $ holds true for any $e \in Win_a (p, \emptyset)_d^s $ and therefore resolves this issue.
\\
The changes of the weak sprectroscopy game impact the definition of strategy formulas. As shown in the proof 
of lemma winning\_budget\_implies\_strategy\_formula in section \ref{derivation:lemma3} our definition that is close 
to the paper still works. (For example one could choose to add a derivation rule corresponding to the move empty\_stbl\_conj\_answer to the definition 
of strategy formulas. We instead constructed the necessary formula in the proof and used that for an empty set the rule stable\_conj could be applied.
More detail is found in the proof in the case $Q=\emptyset$ in case 5 ($g$ being a defender stable conjunction position) 
of case 3 ($g$ being a defender position with successors).)
\item In lemma strategy\_formulas\_distinguish in Section \ref{derivation:lemma3} the case of $g$ being a defender branching position was adjusted: 
When considering $g=(p,\alpha ,p', Q- Q \alpha, Q \alpha)_d^\eta$ the conditions $p \overset{\alpha}{\longrightarrow} p'$ and $Q \alpha \neq \emptyset$ were 
added. 
\end{itemize}
\section{Conclusion}
We were able to formalize the majority of the paper, including the spectroscopy energy game as introduced by Benjamin Bisping and David N. Jansen in \cite{bisping2023lineartimebranchingtime}, 
and prove one direction of the theorem stating correctness, namely ``if the attacker wins the weak spectroscopy game, given an energy $e$, then there exists a formula $\varphi \in$ HML$_{SRBB}$ with price expr$(\varphi) \leq$ $e$''. 
For the other direction, we provide a comprehensive proof skeleton, including proofs for individual induction cases.
\\
Due to the nature of Isabelle, the implementation differs in subtle and some not-so-subtle ways from \cite{bisping2023lineartimebranchingtime}. The gravest change is to the definition of HML$_{SRBB}$. 
We have implemented this definition using three mutually recursive data types. As a result, we had two definitions for a conjunction $\bigwedge\psi$, \textbf{ImmConj} and \textbf{Conj}, each with a different type. 
The other difference to the HML$_{SRBB}$ definition of \cite{bisping2023lineartimebranchingtime} concerned the observation of actions. 
We argue that both definitions have the same distinguishing power. 
These changes led to necessary adaptations of our definition of the weak spectroscopy game and thereby affected the following definitions and proofs.
These and other deviations are usually justified directly at their occurrence in the code. An overview can be found at (Index of changes). 
\\
One major change to \cite{bisping2023lineartimebranchingtime} is a new game move $(p,\varnothing)_{d}^{s}$ $\overset{\hat{e}_4}{\rightarrowtail} (p,\varnothing)_d$. 
Before, the attacker could use an empty stability conjunction \textbf{StableConj} without having the proper budget. Now the defender can exploit that by invoking the new move.
\\
We formalized an energy game closely related to \cite{bisping2023lineartimebranchingtime} that can decide (almost) all behavioral equivalences between stability-respecting branching bisimilarity and weak trace equivalence at once.
Provided our definition of coordinates in an 8-dimensional space of energies corresponds to these equivalences, we implemented a (mostly) machine-checkable proof for the correctness of this spectroscopy game.

\bibliographystyle{abbrv}
\bibliography{root}

\newpage
\appendix
\section{Index of Changes}
\begin{itemize}
    \item In \ref{deviation:eneg} energies are extended to include a negative inhabitant. 
    \item In our definition of the spectroscopy game we changed the conditions of the observation move. 
    Instead of $p \overset{\alpha}{\longrightarrow}p'$, $Q \overset{\alpha}{\longrightarrow} Q'$ and $\alpha \neq \tau$
    we allow an observation move if  $p \overset{(\alpha)}{\longrightarrow}p'$, $Q \overset{(\alpha)}{\longrightarrow} Q'$.
    This is necessary due to our formalization of HML_{SRBB}. We believe this to have the same distinguishing power 
    since observations only occur after internal behaviour. However we omitted a proof. 
    This changes results in a similar change of the corresponding derivation rule for strategy formulas.
    \item In lemma winning\_budget\_implies\_strategy\_formula in \ref{derivation:lemma2} a different kind of induction is used compared to the paper 
    where we only look at direct successors and therefore added the not too interesting case of g being an 
    attacker branching position which was omitted in the paper.
   \item In lemma strategy\_formulas\_distinguish in  \ref{derivation:lemma3} we added a case to account for the difference between 
conjunctions and immediate conjunctions. And due to our formalization when considering a
defender branching position $g=(p,\alpha ,p', Q - Q\alpha, Q\alpha)_d^\eta$ the condition 
$Q \alpha \subseteq Q$ is implicitly given.
  \item Since in our HML we distinguish between a conjunction and an immediate conjunction, in section \ref{stratFormula} 
we added a new strategy formula for immediate conjunction to the definition.
\end{itemize}

\input{Example_Instantiation}
\input{Weak_Traces}

\end{document}
