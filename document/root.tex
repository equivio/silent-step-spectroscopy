\documentclass[11pt,a4paper]{article}
\usepackage[T1]{fontenc}
\usepackage{isabelle,isabellesym}
\usepackage{amsmath}
\usepackage{amssymb}
\usepackage{geometry}
\geometry{
  left=3cm,
  right=3cm,
  top=2cm,
  bottom=4cm,
  bindingoffset=5mm
}

% this should be the last package used
\usepackage{pdfsetup}

% do not indent paragraphs
\setlength\parindent{0pt}

% urls in roman style, theory text in math-similar italics
\urlstyle{rm}
\isabellestyle{it}

\definecolor{keyword}{RGB}{0,153,102}
\definecolor{command}{RGB}{0,102,153}
\isabellestyle{tt}
\renewcommand{\isacommand}[1]{\textcolor{command}{\textbf{#1}}}
\renewcommand{\isakeyword}[1]{\textcolor{keyword}{\textbf{#1}}}


\begin{document}

\title{Linear-Time--Branching-Time Spectroscopy Accounting for Silent Steps}
\author{Lisa Annett Barthel \and Leonard Moritz Hübner \and Caroline Lemke \and Karl Parvis Philipp Mattes \and Lenard Mollenkopf}
\maketitle

\begin{abstract}
We provide an Isabelle/HOL formalization of Bisping and Jansen's \cite{bisping2023lineartimebranchingtime} weak spectroscopy game, which can be used to decide
a range of behavioral equivalences simultaneously, spanning from stability-respecting branching bisimilarity to weak trace equivalence.
We outline the proof of correctness by relating "attacker-winning energy budgets and distinguishing sub 
languages of Hennessy–Milner logic characterized by eight dimensions of formula expressiveness". 
We provide small but necessary adaptations to the game and the proof.
\end{abstract}

\tableofcontents

% include generated text of all theories
\newpage
\section{Introduction}
In our project we are interested in comparing model and specification.
We formalize both as processes and use various techniques to check whether the model fulfills its specification.
One possibility to compare model and specification are games that provide a correspondence between the winner of the game and the behavior of the processes.
Otherwise, we could use a logic, known as Hennessy-Milner logic (HML), to describe formulas that distinguish processes and thus the model and its specification.
These techniques for checking whether the model fulfills its specification may also be compared directly.
The attacker wins a play if and only if there is a formula that distinguishes the corresponding processes.
\\\\
Based on the paper \cite{bisping2023lineartimebranchingtime}, we analyze a game in which moves may have costs and a variant of HML.
We check whether these two techniques actually lead to identical results and thus to the same conclusion on whether the processes have the same behavior.
For this purpose, we examine how to formalize the game, HML and their correspondence with Isabelle and prove their correctness.
\\\\
First, we formalize HML and a subset of HML, which we denote HML$_\text{srbb}$, as well as a price function for formulas.
We then describe our formalization of the spectroscopy game and its basis in the form of energy games.
Following these fundamentals, we outline a proof of correctness for the correspondence between the results of the spectroscopy game and distinguishing formulas.
Finally, we discuss the errors and inaccuracies we found in the paper and thus present our contributions to the paper\cite{bisping2023lineartimebranchingtime}.
\input{LTS}
\input{HML}
\input{HML_SRBB}
\input{Energy}
\input{Expressiveness_Price}
\input{Energy_Games}
\input{Spectroscopy_Game}
\input{Silent_Step_Spectroscopy}
\input{Distinction_Implies_Winning_Budgets}
\input{Strategy_Formulas}
\section{Contributions to the Paper}

\begin{itemize}
\item In lemma strategy\_formulas\_distinguish in \ref{derivation:lemma3} the case of $g$ being a defender branching position was adjusted: 
When considering $g=(p,\alpha ,p', Q- Q \alpha, Q \alpha)_d^\eta$ the conditions $p \overset{\alpha}{\longrightarrow} p'$ and $Q \alpha \neq \emptyset$ were 
added. 
\end{itemize}
\newpage
\section{Conclusion}
We were able to formalize the majority of the paper, including the weak spectroscopy game as introduced by Bisping and Jansen in \cite{bisping2023lineartimebranchingtime}, 
and to prove one direction of the theorem stating correctness, namely 'if the attacker wins the weak spectroscopy game, given an energy $e$, then there exists a formula $\varphi \in$ HML$_\text{SRBB}$ with price expr$(\varphi) \leq$ $e$'(c.f. \cite[lemma $2$, $3$]{bisping2023lineartimebranchingtime}). 
For the other direction, we provide a comprehensive proof skeleton, including proofs for individual induction cases.
\\\\
Due to the nature of Isabelle, the formalization differs from \cite{bisping2023lineartimebranchingtime}. The gravest change is to the definition of HML$_\text{SRBB}$. 
We have implemented this definition using three mutually recursive data types. As a result, we had two definitions for a conjunction $\bigwedge\psi$, 
\texttt{ImmConj} and \texttt{Conj}, each with a different type. 
The other difference to the HML$_\text{SRBB}$ definition of \cite{bisping2023lineartimebranchingtime} concerns the observation of actions. 
We argue that both definitions have the same distinguishing power. 
These changes led to necessary adaptations of our definition of the weak spectroscopy game and thereby affected the following definitions and proofs.
An overview of these and other deviations can be found in appendix \ref{index}. 
\\
A major change compared to \cite{bisping2023lineartimebranchingtime} is the addition of new game move $(p,\emptyset)_{d}^{s}$ $\overset{\hat{e}_4}{\rightarrowtail} (p,\emptyset)_d$ from \texttt{Defender\_Stable\_Conj} to \texttt{Defender\_Conj} if $Q = \emptyset$. 
Without this move, the attacker could use an empty stability conjunction \texttt{StableConj} without having the proper budget. 
We formalized a weak spectroscopy game closely related to \cite{bisping2023lineartimebranchingtime} that can decide (almost) all behavioural equivalences between stability-respecting branching bisimilarity and weak trace equivalence at once.
Provided our definition of energies as eight-dimensional vectors corresponds to these equivalences, we implemented a (mostly) machine-checkable proof for the correctness of this spectroscopy game.
\\\\
To further increase confidence in the results of \cite{bisping2023lineartimebranchingtime}, additional proofs are neccesary. Firstly, the proof for 'given an energy $e$, if there exists a formula $\varphi \in$ HML$_\text{SRBB}$ with price expr$(\varphi) \leq$ $e$, then the attacker wins the weak spectroscopy game' (c.f. \cite[lemma $1$]{bisping2023lineartimebranchingtime}).
Secondly, \cite{bisping2023lineartimebranchingtime} uses coordinates of energies to define equivalences. One can show that the HML sublanguages obtained from these coordinates correspond to the desired equivalences.
Since our formalization of the model relation \texttt{hml\_models} is only defined on the parameterization of HML by the state type \texttt{'s}, one could also show that this formalization sufficiently captures the expressiveness power of HML on labelled transition systems.
Finally, \cite[proposition 1]{bisping2023lineartimebranchingtime} claims that their slightly different modal characterization of HML$_\text{SRBB}$ corresponds to the modal characterization of \cite{FOKKINK2019104435}. 
The proof for proposition $1$ in \cite{bisping2023lineartimebranchingtime} could be turned into a machine-checkable proof. 

\bibliographystyle{abbrv}
\bibliography{root}

\newpage
\appendix
\section{Index of Changes}

\begin{itemize}
    \item  In lemma winning\_budget\_implies\_strategy\_formula in \ref{derivation:lemma2} a different kind of induction is used compared to the paper 
where we only look at direct successors and therefore added the not too interesting case of g being an 
attacker branching position which was omitted in the paper.
\end{itemize}

\input{Example_Instantiation}
\input{HML_Context}
\input{Weak_Traces}

\end{document}
