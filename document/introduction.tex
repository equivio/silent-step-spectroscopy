\newpage
\section{Introduction} %todo reference to expose
Verification and asking wether a model fulfills its specification or if a program can be replaced with one that has the same behaviour are core problems of reactive systems and programming.
For this we have to get an idea of what same behavior for processes actually means.
One possibility are games where the winner of the game corresponds to the behavioural equivalence of processes.
Alternatively, we could use a modal logic, known as Hennessy-Milner Logic (HML), not only to express the specification but also to build formulas that distinguish processes and 
thereby characterize behavioural equivalences.
These techniques for checking whether two processes have the same behaviour may also be combined.
\\\\
Previously, it was only possible to decide equivalence problems individually, but recently there have been ideas of deciding many of these problems at once.
Therefore, Bisping and Jansen \cite{bisping2023lineartimebranchingtime} included a measure of expressiveness for HML$_\text{{SRBB}}$ formulas as eight-dimensional vectors.
These vectors are added as costs to the moves of an extended delay bisimulation game such that the following property is obtained: 
The attacker wins a play with a certain initial energy $e$ if and only if there is a formula that distinguishes the corresponding processes with a price less than or equal to $e$.
Then the initial energy and the price of a formula encode the satisfied behavioural equivalences.
Therefore it is possible to decide for a whole spectrum of behavioural equivalences at the same time which of them apply.
\\\\
We formalize the eight-dimensional spectroscopy energy game in which moves may have costs and a variant of HML introduced by Benjamin Bisping and David N. Jansen in \cite{bisping2023lineartimebranchingtime}, 
which ``can be used to decide a wide array of behavioral equivalences between stability-respecting branching bisimilarity and weak trace equivalence in one go''\cite{bisping2023lineartimebranchingtime}.
We then prove the correspondence between ``attacker-winning energy budgets and distinguishing sublanguages of Hennessy-Milner logic characterized by eight dimensions of formula expressiveness'' 
and thus the identical results of these two techniques as to whether the processes have the same behaviour.
With our formalization, we try to follow \cite{bisping2023lineartimebranchingtime} as closely as possible in how we formalize the spectroscopy game, HML and their correspondence.
In doing so, we point out deviations in our formalization from the paper as well as errors that we were able to find and which we have corrected.
\\\\
First, we formalize the term LTS and the different kinds of steps.
Afterwards we describe our formalization of HML and a subset of HML, which we denote HML$_\text{srbb}$.
Within this HML section, we introduce several implications and equivalences that we expect to hold for HML formulas.
In the following sections, we present our formalization of energies as a data type that should fulfill certain properties and a price function for formulas.
Before we outline a proof of correctness for the correspondence between the results of the spectroscopy game and distinguishing formulas we formalize spectroscopy game and its basis in the form of energy games.
Following these fundamentals, we state our formalization of the theorem $1$ of \cite{bisping2023lineartimebranchingtime}, that ``relates attacker-winning energy budgets and distinguishing sublanguages of Hennessy-Milner logic''. 
Based on the proof in \cite{bisping2023lineartimebranchingtime} we show this theorem through three lemmas. 
The first lemma states that given a distinguishing formula, the attacker is able to win the corresponding spectroscopy game, which we prove by induction.
Using the same technique, we prove the second lemma, which claimes that if the attacker wins the spectroscopy game with an initial energy $e$, then there exists a (strategy) formula with a price less than or equal to $e$.
Afterwards, the third lemma completes this cycle by stating that if there is a (strategy) formula $\phi$, then $\phi$ distingusihes the corresponding processes.
Finally, we discuss the errors and inaccuracies we found in the paper and thus present our contributions to the paper\cite{bisping2023lineartimebranchingtime}.