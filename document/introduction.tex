\newpage
\section{Introduction}
In our project we are interested in comparing model and specification.
We formalize both as processes and use various techniques to check whether the model fulfills its specification.
One possibility to compare model and specification are games that provide a correspondence between the winner of the game and the behavior of the processes.
Otherwise, we could use a logic, known as Hennessy-Milner logic (HML), to describe formulas that distinguish processes and thus the model and its specification.
These techniques for checking whether the model fulfills its specification may also be compared directly.
The attacker wins a play if and only if there is a formula that distinguishes the corresponding processes.
\\\\
Based on the paper \cite{bisping2023lineartimebranchingtime}, we formalize a game in which moves may have costs and a variant of HML.
We check whether these two techniques actually lead to identical results and thus to the same conclusion on whether the processes have the same behavior.
For this purpose, we examine how to formalize the game, HML and their correspondence with Isabelle and prove their correctness.
\\\\
First, we formalize HML and a subset of HML, which we denote HML$_\text{srbb}$, as well as a price function for formulas.
We then describe our formalization of the spectroscopy game and its basis in the form of energy games.
Following these fundamentals, we outline a proof of correctness for the correspondence between the results of the spectroscopy game and distinguishing formulas.
Finally, we discuss the errors and inaccuracies we found in the paper and thus present our contributions to the paper\cite{bisping2023lineartimebranchingtime}.