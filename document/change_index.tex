\section{Index of Changes}
\begin{itemize}
    \item In \ref{deviation:lemma1_Q_a} the sixt case of the inductiv property is ammended by $Q_\alpha \neq {}$,
    in order to be able to directly apply the induction hypothesis in the proof.
    \item In \ref{deviation:lemma1TT} the proof of lemma 1 has to deal with formulas of the form \texttt{TT}. 
    \item  In lemma winning\_budget\_implies\_strategy\_formula in \ref{derivation:lemma2} a different kind of induction is used compared to the paper 
    where we only look at direct successors and therefore added the not too interesting case of g being an 
    attacker branching position which was omitted in the paper.
   \item In lemma strategy\_formulas\_distinguish in  \ref{derivation:lemma3} we added a case to account for the difference between 
    conjunctions and immediate conjunctions. And due to our formalization when considering a
    defender branching position $g=(p,\alpha ,p', Q \ Q\alpha, Q\alpha)_d^\eta$ the condition 
    $Q \alpha \subseteq Q$ is implicitly given.
\end{itemize}
