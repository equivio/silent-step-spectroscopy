\section{Index of Changes}
\begin{itemize}
    \item In \ref{deviation:eneg} energies are extended to include a negative inhabitant. 
    \item In \cite{bisping2023lineartimebranchingtime} the winning budget of a state in an energy game is defined as the set of all starting energies such that there exists a winning strategy for the attacker when starting from that state. Our formalization does not define attacker winning strategies and instead uses the characterization by proposition 2 \cite[p. 9]{bisping2023lineartimebranchingtime} as the definition of winning budgets.
and instead used proposition 2 as a definition of winning budgets. We also omitted an explicit definition of winning a game and implicitly assumed that there 
always exists a winner and the attacker wins if and only if the starting energy is in the winning budget of the starting position. This holds by declaring all
infinite plays to be won by the defender. A proof is given in the paper. 
    \item In our definition of the spectroscopy game we changed the conditions of the observation move. 
    Instead of $p \overset{\alpha}{\longrightarrow}p'$, $Q \overset{\alpha}{\longrightarrow} Q'$ and $\alpha \neq \tau$
    we allow an observation move if  $p \overset{(\alpha)}{\longrightarrow}p'$, $Q \overset{(\alpha)}{\longrightarrow} Q'$.
    This is necessary due to our formalization of $HML_{SRBB}$. We believe this to have the same distinguishing power 
    since observations only occur after internal behaviour. However we omitted a proof. 
    This results in a similar change of the corresponding derivation rule for strategy formulas.
    \item In lemma winning\_budget\_implies\_strategy\_formula in \ref{derivation:lemma2} a different kind of induction is used compared to the paper 
    where we only look at direct successors and therefore added the not too interesting case of g being an 
    attacker branching position which was omitted in the paper.
   \item In lemma strategy\_formulas\_distinguish in  \ref{derivation:lemma3} we added a case to account for the difference between 
conjunctions and immediate conjunctions. And due to our formalization when considering a
defender branching position $g=(p,\alpha ,p', Q - Q\alpha, Q\alpha)_d^\eta$ the condition 
$Q \alpha \subseteq Q$ is implicitly given.
  \item Since in our HML we distinguish between a conjunction and an immediate conjunction, in section \ref{stratFormula} 
we added a new strategy formula for immediate conjunction to the definition.
\end{itemize}
