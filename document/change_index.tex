\section{Index of Changes}
\begin{itemize}
    \item In the definition of \texttt{hml\_srbb} (in section \ref{sect:hmlSRBB}), we change the inner observation from
      $\langle\alpha\rangle\varphi$ with side condition $\alpha \neq \tau$ to $(\alpha)\varphi$.
      This was done due to an inability to exclude $\tau$ from the label type while defining \texttt{hml\_srbb}. 
    \item In \ref{deviation:eneg} energies are extended to include a negative inhabitant. 
    \item In our definition of the spectroscopy game we changed the conditions of the observation move. 
    Instead of $p \overset{\alpha}{\longrightarrow}p'$, $Q \overset{\alpha}{\longrightarrow} Q'$ and $\alpha \neq \tau$
    we allow an observation move if  $p \overset{(\alpha)}{\longrightarrow}p'$, $Q \overset{(\alpha)}{\longrightarrow} Q'$.
    This is necessary due to our formalization of $HML_{SRBB}$. We believe this to have the same distinguishing power 
    since observations only occur after internal behaviour. However we omitted a proof. 
    This results in a similar change of the corresponding derivation rule for strategy formulas.
    \item In lemma winning\_budget\_implies\_strategy\_formula in \ref{derivation:lemma2} a different kind of induction is used compared to the paper 
    where we only look at direct successors and therefore added the not too interesting case of g being an 
    attacker branching position which was omitted in the paper.
   \item In lemma strategy\_formulas\_distinguish in  \ref{derivation:lemma3} we added a case to account for the difference between 
conjunctions and immediate conjunctions. And due to our formalization when considering a
defender branching position $g=(p,\alpha ,p', Q - Q\alpha, Q\alpha)_d^\eta$ the condition 
$Q \alpha \subseteq Q$ is implicitly given.
  \item Since in our HML we distinguish between a conjunction and an immediate conjunction, in section \ref{stratFormula} 
we added a new strategy formula for immediate conjunction to the definition.
\end{itemize}
