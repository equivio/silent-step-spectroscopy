\newpage
\section{Contributions to the Paper}
In this section, we present our outcomes and contributions to the paper.
\begin{itemize}
    \item In the definition of the weak spectroscopy game in section \ref{specmoves} the moves were changed: 
    The move `finishing' was removed. This move adjusted the cost of an empty immediate conjunction to be zero.
    This is inconsistent with the definition of the expressiveness price. This move is also not needed, as in the case of $Q$ being an empty set at an attacker immediate position, the attacker can move to the corresponding
    defender conjunction position by performing a `delay' move, followed by a `late instable conjunction' move. Both of these moves have zero costs.

    \item Another change in the definition of the weak spectroscopy game in section \ref{specmoves} is the addition of the move \texttt{empty\_stbl\_conj\_answer} (moving from a defender stable conjunction position to a defender conjunction
    position if $Q$ is empty) with costs $e_4$. This is necessary for the proof of lemma\\ 
    \texttt{winning\_budget\_implies\_strategy\_formula} in section \ref{deviation:lemma3}. Before 
    $e \in \text{Win}_a (p, \emptyset)_d^s $ was true for any energy (not equal to the \texttt{defender\_win\_level}, in particular $\bold{0}$. 
    Since an empty stable conjuction has costs of $e_4$, no strategy formula with an appropriate price could be constructed. 
    With the added move $e_4 \leq e $ holds true for any $e \in \text{Win}_a (p, \emptyset)_d^s $ and therefore resolves this issue.

    \item The proof of \texttt{distinction\_implies\_winning\_budgets} in section \ref{deviation:lemma1_Q_a} 
    (c.f. \cite[lemma $1$]{bisping2023lineartimebranchingtime}) required ammending, the sixth case
    of the inductive property is ammended by $Q_\alpha \neq {}$, in order to be able to directly
    apply the induction hypothesis in the proof.

    \item The sixth inductive property of the lemma \texttt{strategy\_formulas\_distinguish} in section \ref{deviation:lemma3} 
    (c.f. \cite[lemma $3$]{bisping2023lineartimebranchingtime}) was adjusted.
    The case of $g=(p,\alpha ,p', Q- Q_\alpha, Q_\alpha)_d^\eta$ being a defender branching position
    required adding the assumtions that $p \overset{\alpha}{\longrightarrow} p'$ and $Q_\alpha \neq \emptyset$
    to match those of the branching conjunction move rule (cf. \cite[p. 13]{bisping2023lineartimebranchingtime}). 

    \item In lemma \texttt{strategy\_formulas\_distinguish} in section \ref{deviation:lemma3} the case of $g$ being a defender branching position was adjusted: 
    
    When considering $g=(p,\alpha ,p', Q- Q \alpha, Q \alpha)_d^\eta$ the conditions $p \overset{\alpha}{\longrightarrow} p'$ and $Q \alpha \neq \emptyset$ were added. 
\end{itemize}
