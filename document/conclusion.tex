\section{Conclusion}
We were able to formalize the majority of the paper, including the spectroscopy energy game as introduced by Benjamin Bisping and David N. Jansen in \cite{bisping2023lineartimebranchingtime}, 
and prove one direction of the theorem stating correctness, namely ``if the attacker wins the weak spectroscopy game, given an energy $e$, then there exists a formula $\varphi \in$ HML$_{SRBB}$ with price expr$(\varphi) \leq$ $e$''. 
For the other direction, we provide a comprehensive proof skeleton, including proofs for individual induction cases.
\\
Due to the nature of Isabelle, the implementation differs in subtle and some not-so-subtle ways from \cite{bisping2023lineartimebranchingtime}. The gravest change is to the definition of HML$_{SRBB}$. 
We have implemented this definition using three mutually recursive data types. As a result, we had two definitions for a conjunction $\bigwedge\psi$, \textbf{ImmConj} and \textbf{Conj}, each with a different type. 
The other difference to the HML$_{SRBB}$ definition of \cite{bisping2023lineartimebranchingtime} concerned the observation of actions. 
We argue that both definitions have the same distinguishing power. 
These changes led to necessary adaptations of our definition of the weak spectroscopy game and thereby affected the following definitions and proofs.
These and other deviations are usually justified directly at their occurrence in the code. An overview can be found at (Index of changes). 
\\
One major change to \cite{bisping2023lineartimebranchingtime} is a new game move $(p,\varnothing)_{d}^{s}$ $\overset{\hat{e}_4}{\rightarrowtail} (p,\varnothing)_d$. 
Before, the attacker could use an empty stability conjunction \textbf{StableConj} without having the proper budget. Now the defender can exploit that by invoking the new move.
\\
We formalized an energy game closely related to \cite{bisping2023lineartimebranchingtime} that can decide (almost) all behavioral equivalences between stability-respecting branching bisimilarity and weak trace equivalence at once.
Provided our definition of coordinates in an 8-dimensional space of energies corresponds to these equivalences, we implemented a (mostly) machine-checkable proof for the correctness of this spectroscopy game.